% This is the separate abstract for JISA 2017
% eija checked this 24062018
% grammarly lute checked 18 Jan 2018

\begin{abstract} % abstract
The dictum of ``Release early, release often.'' by Eric Raymond as the Linux modus operandi highlights the importance of release management in open source software development.  
However, there are very few empirical studies addressing release management in this context. 
It is already known that most open source software communities adopt either a feature-based or time-based release strategy. 
Both have their own advantages and disadvantages that are also context-specific. Recent research reports that many prominent open source software projects have overcome a number of recurrent problems by moving from feature-based to time-based release strategies.
In this longitudinal case study, we address the release management practices of OpenStack, a large scale open source project developing cloud computing technologies. We discuss how the release management practices of OpenStack have evolved in terms of chosen strategy and timeframes, and with close attention to processes and tools. We discuss the number of practical and managerial issues related to release management within the context of large and complex software ecosystems. Our findings also reveal that multiple release management cycles can co-exist in large and complex software ecosystems such as OpenStack. 
\end{abstract}

%%%%%%%%%%%%%%%%%%%%%%%%%%%%%%%%%%%%%%%%%%%%%%
%%                                          %%
%% The keywords begin here                  %%
%%                                          %%
%% Put each keyword in separate \kwd{}.     %%
%%                                          %%
%%%%%%%%%%%%%%%%%%%%%%%%%%%%%%%%%%%%%%%%%%%%%%

\begin{keyword}
 \kwd{Open-Source}
 \kwd{OSS}
 \kwd{FLOSS}
 \kwd{Release Management}
 \kwd{Release Engineering}
 \kwd{OpenStack}
\end{keyword}